\documentclass[addpoints]{exam}

\usepackage{amsmath,amssymb,amsthm}
\usepackage{tabularx}

% \usepackage{draftwatermark}
% \SetWatermarkText{Sample Solution}
% \SetWatermarkScale{3}
% \printanswers

\theoremstyle{definition}
% Option [section] can be added to refer to the section number in the definition numbering.
\newtheorem{definition}{Definition}

\theoremstyle{claim}
\newtheorem{claim}{Claim}

\runningheader{CS/MATH 113 L1}{Quiz 1A}{spring 2024}
\runningheadrule
\runningfootrule
\runningfooter{}{Page \thepage\ of \numpages}{}

\title{Quiz 1A: Even Numbers}
\author{CS/MATH 113 Discrete Mathematics L1}
\date{Habib University | Spring 2024}

\begin{document}
\maketitle
\thispagestyle{empty}
\noindent
\begin{tabularx}{\linewidth}{Xr}
  Total Marks: \numpoints & Date: \today\\
  Duration: 10 minutes & Time: 830--840h
\end{tabularx}
\hrule
\bigskip

\noindent \textbf{Student ID}: \hrulefill \\[5pt]
\noindent \textbf{Student Name}: \hrulefill \\[5pt]

\begin{questions}
  \question[10] Given the following definition, prove or disprove the claim below.

  % Does anyone have a clean way to make this definition number compile as 1 instead of 0.1?
  % Yes. The [section] argument is removed from the definition above.
  \begin{definition}[Even numbers]
    An integer is \textit{even} if it can be written as $2k$ for some $k\in\mathbb{Z}$.
  \end{definition}

  \begin{claim}
    The sum of two even numbers is even.
  \end{claim}

  \begin{solution}
    We attempt a direct proof using the definition of even numbers.
    \begin{proof}
      Let the numbers be $x$ and $y$.\\
      Then $x=2i$ and $y=2j$ for some $i,j\in\mathbb{Z}$.\\
      Then $x+y=2i+2j=2(i+j)$.
    \end{proof}
  \end{solution}
\end{questions}
\end{document}

%%% Local Variables:
%%% mode: latex
%%% TeX-master: t
%%% End:
